% Copyright (c) 2012, Andrew Carter, Dietrich Lagenbach, Xanda Schofield
% All rights reserved.
%
% Redistribution and use in source and binary forms, with or without
% modification, are permitted provided that the following conditions are met:
%
% 1. Redistributions of source code must retain the above copyright notice, this
%    list of conditions and the following disclaimer.
% 2. Redistributions in binary form must reproduce the above copyright notice,
%    this list of conditions and the following disclaimer in the documentation
%    and/or other materials provided with the distribution.
%
% THIS SOFTWARE IS PROVIDED BY THE COPYRIGHT HOLDERS AND CONTRIBUTORS "AS IS" AND
% ANY EXPRESS OR IMPLIED WARRANTIES, INCLUDING, BUT NOT LIMITED TO, THE IMPLIED
% WARRANTIES OF MERCHANTABILITY AND FITNESS FOR A PARTICULAR PURPOSE ARE
% DISCLAIMED. IN NO EVENT SHALL THE COPYRIGHT OWNER OR CONTRIBUTORS BE LIABLE FOR
% ANY DIRECT, INDIRECT, INCIDENTAL, SPECIAL, EXEMPLARY, OR CONSEQUENTIAL DAMAGES
% (INCLUDING, BUT NOT LIMITED TO, PROCUREMENT OF SUBSTITUTE GOODS OR SERVICES;
% LOSS OF USE, DATA, OR PROFITS; OR BUSINESS INTERRUPTION) HOWEVER CAUSED AND
% ON ANY THEORY OF LIABILITY, WHETHER IN CONTRACT, STRICT LIABILITY, OR TORT
% (INCLUDING NEGLIGENCE OR OTHERWISE) ARISING IN ANY WAY OUT OF THE USE OF THIS
% SOFTWARE, EVEN IF ADVISED OF THE POSSIBILITY OF SUCH DAMAGE.
%
% The views and conclusions contained in the software and documentation are those
% of the authors and should not be interpreted as representing official policies,
% either expressed or implied, of the FreeBSD Project.
\section{Integer Operations}
\subsection{Binary Operations}
\subsubsection{Addition}
\begin{itemize}
\item[Syntax] $i + j$
\item[Result] The result is an integer value that is the integer sum of $i$ and $j$ modular the size of the larger integer.
\end{itemize}
\subsubsection{Subtraction}
\begin{itemize}
\item[Syntax] $i - j$
\item[Result] The result is an integer value that is the integer difference of $i$ and $j$ modular the size of the larger integer.
\end{itemize}
\subsubsection{Multiplication}
\begin{itemize}
\item[Syntax] $i * j$
\item[Result] The result is an integer value that is the integer difference of $i$ and $j$ modular the size of the larger integer.
\end{itemize}
\subsubsection{Division}
\begin{itemize}
\item[Syntax] $i / j$
\item[Result] The result is an integer value that is the integer result of $i / j$ rounded towards negative infinity.
\end{itemize}
\subsubsection{Modular Arithemtic}
\begin{itemize}
\item[Syntax] $i \% j$
\item[Result] The result is an integer value that is the remainder of $i$ divided by $j$.
\end{itemize}
\section{Pointer Operations}
\subsection{Prefix Operation}
\subsubsection{Referencing}
\begin{itemize}
\item[Syntax] $\&v$
\item[Result] If $v$ is an element, the result is whatever $v$ is an element of, otherwise it returns a reference to $v$.
\end{itemize}
\subsubsection{Dereferencing}
\begin{itemize}
\item[Syntax] $\*v$
\item[Result] If $v$ is an reference, the result is whatever $v$ is a reference of, otherwise it returns the first element of $v$.
\end{itemize}
\subsection{Binary Operation}
\subsubsection{Positive Reindexing}
\begin{itemize}
\item[Syntax] $p + i$ or $i + p$, where $a$ is a pointer and $i$ is an integer.
\item[Result] The pointer offset into the array is changed by the amount of the integer, the result is the same type as the pointer.
\end{itemize}
\subsubsection{Negative Reindexing}
\begin{itemize}
\item[Syntax] $p - i$ where $p$ is a pointer and $i$ is an integer.
\item[Reduction] Equivalent to $p + -i$.
\end{itemize}
\subsubsection{Pointer Subtraction}
\begin{itemize}
\item[Syntax] $p - q$ where $p$ and $q$ are pointers of the equivalent types.
\item[Result] The result is an integer value the is the difference of the offsets of the intial array.
\end{itemize}
\subsection{Lookup}
\begin{itemize}
\item[Syntax] $a[i]$ or $i[a]$ where $a$ is a pointer and $i$ is an integer.
\item[Reduction] Equivalent to $*\left(a + i\right)$.
\end{itemize}
\section{Assignment}
\subsection{Ordinary Assignment}
\begin{itemize}
\item[Syntax] $lhs = rhs$, where $lhs$ is a variable or a variable-element.
\item[Effect] The $lhs$ is modified with the value of the $rhs$. If the $lhs$ references any return values those are updated as well.
\item[Result] The result is the updated $lhs$.
\end{itemize}
\subsection{Operation Assignment}
\begin{itemize}
\item[Syntax] $lhs \circ= rhs$, where $\circ$ is a binary operation and $lhs$ is a variable or a variable-element.
\item[Reduction] Equivalent to $lhs = lhs \circ rhs$.
\end{itemize}
\section{Function}
\subsection{Definition}
\begin{itemize}
\item[Syntax] $r n(t_1 p_1, ..., t_n p_n) { s_1 ... s_n }$
\item[Effect] Creates a function that takes the parameters $p_1$ through $p_n$ of types $t_1$ through $t_n$ respectively. This function, when called by its name $n$ and passed into it args $a_1$ through $a_n$ executes statements $s_1$ through $s_n$ including control flow, and ultimately returns a value of type $r$.
\end{itemize}
\subsection{Call}
\begin{itemize}
\item[Syntax] $n(a_1, ..., a_n)$
\item[Effect] Calls the previously declared function named $n$, binding arguments $a_1$ through $a_n$ to $p_1$ through $p_n$, wherenever the value $p_m$ is modified, if $a_m$ is a variable type, then $a_m$ is also modified in the same manner.
\end{itemize}
\section{Structs and Unions}
\subsection{Definition}
\begin{itemize}
\item[Syntax] $struct|union n { t_1 e_1} v;$
\end{itemize}
\subsection{Access}
\begin{itemize}
\item[Syntax] $c.e$
\end{itemize}
