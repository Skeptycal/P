% Copyright (c) 2012, Andrew Carter, Dietrich Langenbach, Xanda Schofield
% All rights reserved.
%
% Redistribution and use in source and binary forms, with or without
% modification, are permitted provided that the following conditions are met:
%
% 1. Redistributions of source code must retain the above copyright notice, this
%    list of conditions and the following disclaimer.
% 2. Redistributions in binary form must reproduce the above copyright notice,
%    this list of conditions and the following disclaimer in the documentation
%    and/or other materials provided with the distribution.
%
% THIS SOFTWARE IS PROVIDED BY THE COPYRIGHT HOLDERS AND CONTRIBUTORS "AS IS" AND
% ANY EXPRESS OR IMPLIED WARRANTIES, INCLUDING, BUT NOT LIMITED TO, THE IMPLIED
% WARRANTIES OF MERCHANTABILITY AND FITNESS FOR A PARTICULAR PURPOSE ARE
% DISCLAIMED. IN NO EVENT SHALL THE COPYRIGHT OWNER OR CONTRIBUTORS BE LIABLE FOR
% ANY DIRECT, INDIRECT, INCIDENTAL, SPECIAL, EXEMPLARY, OR CONSEQUENTIAL DAMAGES
% (INCLUDING, BUT NOT LIMITED TO, PROCUREMENT OF SUBSTITUTE GOODS OR SERVICES;
% LOSS OF USE, DATA, OR PROFITS; OR BUSINESS INTERRUPTION) HOWEVER CAUSED AND
% ON ANY THEORY OF LIABILITY, WHETHER IN CONTRACT, STRICT LIABILITY, OR TORT
% (INCLUDING NEGLIGENCE OR OTHERWISE) ARISING IN ANY WAY OUT OF THE USE OF THIS
% SOFTWARE, EVEN IF ADVISED OF THE POSSIBILITY OF SUCH DAMAGE.
%
% The views and conclusions contained in the software and documentation are those
% of the authors and should not be interpreted as representing official policies,
% either expressed or implied, of the FreeBSD Project.
\section{access}
to read or modify the value of an object
\begin{itemize}
\item[NOTE 1] Where only one of these two actions is meant, ``read'' or ``modify'' is used.
\item[NOTE 2] ``Modify'' includes the case where the value being stored is the same as the previous value.
\item[NOTE 3] Expressions that are not evaluated do not access objects.
\end{itemize}
\section{argument}
expression in the comma-separated list bounded by the parentheses in a function call expression, or a sequence of preprocessing tokens in the comma-separated list bounded by the parentheses in a function-like macro invocation
\section{behavior}
external appearance or action
\subsection{implementation-defined behavior}
unspecified behavior where each implementation documents how the choice is made.
\subsection{undefined behavior}
behavior, upon use of a nonportable or erroneous program construct or of erroneous data,
for which this Standard imposes no requirements.
\section{bit}
unit of data storage in the execution environment large enough to hold an object that may have one of two values
\section{element}
part of a value, variable, reference, or another element. Elements can be split into two catagories, value-elements and variable-elements
\subsection{value-element}
A value-element can be defined inductively as an element of a value, or an element of a value-element
\subsection{variable-element}
A variable-element can be defined inductively as an element of a variable, or an element of a variable-element
\section{forward reference}
reference to a later section of this Standard that contains additional information relevant to this subclause
\section{reference}
a relation to the part or whole of another variable, element, or reference. This is the opposite of an element
\subsection{return-reference}
a reference to an object that acts as an additional return value for a function
\section{parameter}
variable declared as part of a function declaration that aquires a value or return-reference upon entry into the function, or a name denoting a replacement inside a function-like macro
