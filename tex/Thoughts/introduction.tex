% Copyright (c) 2012, Andrew Carter, Dietrich Langenbach, Xanda Schofield
% All rights reserved.
%
% Redistribution and use in source and binary forms, with or without
% modification, are permitted provided that the following conditions are met:
%
% 1. Redistributions of source code must retain the above copyright notice, this
%    list of conditions and the following disclaimer.
% 2. Redistributions in binary form must reproduce the above copyright notice,
%    this list of conditions and the following disclaimer in the documentation
%    and/or other materials provided with the distribution.
%
% THIS SOFTWARE IS PROVIDED BY THE COPYRIGHT HOLDERS AND CONTRIBUTORS "AS IS" AND
% ANY EXPRESS OR IMPLIED WARRANTIES, INCLUDING, BUT NOT LIMITED TO, THE IMPLIED
% WARRANTIES OF MERCHANTABILITY AND FITNESS FOR A PARTICULAR PURPOSE ARE
% DISCLAIMED. IN NO EVENT SHALL THE COPYRIGHT OWNER OR CONTRIBUTORS BE LIABLE FOR
% ANY DIRECT, INDIRECT, INCIDENTAL, SPECIAL, EXEMPLARY, OR CONSEQUENTIAL DAMAGES
% (INCLUDING, BUT NOT LIMITED TO, PROCUREMENT OF SUBSTITUTE GOODS OR SERVICES;
% LOSS OF USE, DATA, OR PROFITS; OR BUSINESS INTERRUPTION) HOWEVER CAUSED AND
% ON ANY THEORY OF LIABILITY, WHETHER IN CONTRACT, STRICT LIABILITY, OR TORT
% (INCLUDING NEGLIGENCE OR OTHERWISE) ARISING IN ANY WAY OUT OF THE USE OF THIS
% SOFTWARE, EVEN IF ADVISED OF THE POSSIBILITY OF SUCH DAMAGE.
%
% The views and conclusions contained in the software and documentation are those
% of the authors and should not be interpreted as representing official policies,
% either expressed or implied, of the FreeBSD Project.
\section{Purpose}
This document is designed to detail possible semantic evaluation of a P Program, possible optimizations that the compiler can do, as well as various philosphies or 
beliefs that support these.
\section{Basic Philosophy}
Code optimization and readability are currently a tradeoff against each other. It seems unreasonable that this should always be the case, it seems the clearer one can 
express what one is doing to other people, the clearer one can express to a computer what needs to be done. The first problem is that in C, C++, Java etc. we are not 
really expressing what needs to be done, but rather what the computer should be doing. Computers now are complex beasts with levels of caches, increasing number of cores, 
and predictors. Optimizing for them is going to lead to a complex set of instrutions, so it makes sense that the code is going to get gnarly. Humans aren't very good at 
it either, there are obviou optimizations, but getting stuff to fit properly in cache is trial and error, even if we can't give a computer enough information to make it 
faster surely we can teach it trial and error.

Functional programming languages does treat a program more as a description of what needs to be done. It still has heiarchy, and gives a way to compute the data, but its 
a little bit more flexible, its easier to reorder things, results from one function call do not interfere with other indpenendent function calls (mostly). Obviously a 
first stab at functional languages, unless heavily specialized (i.e. APL) are going to be slower. There is overhead in converting between a functional program and the 
assembly code that a computer requires. In addition programmers often find writing in these languages unintuitive, which admitedly may be because they were brought up 
learning languages like C,C++, and Java.

The big semantic difference (the syntax is obviously different, so its a bit of a leap) is the concept of aliasing. A Let block is very similar to an assignment 
statement, but barring impurities (setf) there is effectively a deep copy, or no aliasing, admittedly Haskell which prides itself more on its purity may have been a 
better example but hopefully the point is made. So what if there was a programming language that looked like C but acted like a functional programming language? Obviously 
there are some differences, global variables, sometime functions take in arguments for the purpose of replacing them, prevalance of arrays in C, existense of lists/tuples 
in Lisp, etc. so it is not a straight forward conversion. This is what P is supposed to be. I'm hoping that the language can be designed such that a lowlevel C program 
can be programmed in good style in C act the same as the same code (I'll allow slight modification) in P, and get the P compiler to recognize that it should be the same 
as the lowlevel C program. I happen to prefer soccer to basketball (goals are reached less often in soccer).
\section{Premise}
The P Programming language is a new programming language. It is based on the syntax of C, however with different semantics. The key difference is that pointer assignment 
operations appear to the programmer as a deep copy. Furthermore pointers passed into functions do no alias with each other from the perspective of the function being 
called.
\section{Approach}
The approach to creating P is as follows. First identify the problems with the conversion, write sample code and explain semantics for possible solutions to show the 
strengths or weaknesses of various interpretations. Come up with some consensus, document it and begin writing the compiler of a subset of P that shows the behavior, 
revise consensus if necessary. Once we have a working compiler we can begin working in two different directions, extending the compiler and begin optimizing the output. 
It will be important to keep track of what we want to be doing, and what the compiler is currently doing, and who is currently working on making those two closer to each 
other. Documenting additions will be just as important as adding them.

